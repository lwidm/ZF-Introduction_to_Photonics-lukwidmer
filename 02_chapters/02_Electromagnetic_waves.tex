%!TEX root = ../ZF-Introduction_to_Photonics-lukwidmer.tex
\mySection{Electromagnetic waves}
\vspace{2mm}
\normalsize\textbf{Wave equation for electromagnetic waves}\small
\hrule
\begin{align*}
	\nabla^2 \vv{E} - \mu_0\varepsilon_0 \frac{\partial^2 \vv{E}}{\partial t^2} = 0 \\
	\nabla^2 \vv{B} - \mu_0\varepsilon_0 \frac{\partial^2 \vv{B}}{\partial t^2} = 0 \\
\end{align*}

\WhiteSpace

\mySubsection{Maxwell Equations}{3}{
	\vspace{2mm}
	\normalsize\textbf{Gauss' law for electric fields}\small
	\hrule
	\myBox {
		$$\nabla \cdot \vv{E} = \frac{\varrho}{\varepsilon_0}$$
	}
	The number of electric field lines passing through a closed surface is proportional to the electric charge enclosed by the surface is a proportional to the electric charge enclosed by the surface. An isolated positive charge generates a divergent electric field.

	\vspace{2mm}
	\normalsize\textbf{Gauss' law for magnetic fields}\small
	\hrule
	\myBox {
		$$\nabla \cdot \vv{E} = \frac{\varrho}{\varepsilon_0}$$
	}
	There is no isolated "magnetic charge". The magnetic field lines circulate back on themselves.

	\vspace{2mm}
	\normalsize\textbf{Faraday's law}\small
	\hrule
	\myBox {
		$$ \vv{nabla} \times \vv{E} = - \partialdiff{\vv{B}}{t}$$
	}
	A time varying magnetic field generates a circulating electric field.

	\vspace{2mm}
	\normalsize\textbf{Ampère-Maxwell law}\small
	\hrule
	\myBox {
		$$ \nabla \times \vv{B} = \mu_0 \codt \vv{J} + \mu_0 \varepsilon_0 \partialdiff{\vv{E}}{t}$$
	}
	A circulating magnetic field can be generated either by an electric current or by a time varying electric field.


}
